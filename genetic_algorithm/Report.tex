\documentclass[12pt]{article}
\usepackage{design_ASC}
\usepackage{graphicx}
\graphicspath{ {./images/} }
% \usepackage[LGR]{fontenc}
\usepackage[utf8]{inputenc}
\usepackage[greek,english]{babel}
\usepackage{alphabeta}
\usepackage{placeins}
\usepackage{hyperref}
\newcommand\tab[1][1cm]{\hspace*{#1}}
\setlength\parindent{0pt} 

%% -----------------------------
%% TITLE
%% -----------------------------
\title{Part 2} %% Assignment Title

\author{Τρίμας Χρήστος , Παντελής Κωσταντίνος\\%% Student name
2016030054 , 2015030070 \\Τεχνητή Νοημοσύνη\\
LAB41744549\\%% Code and course name
\textsc{Πολυτεχνείο Κρήτης}
}

\date{\today} %% Change "\today" by another date manually
%% -----------------------------
%% -----------------------------

%% %%%%%%%%%%%%%%%%%%%%%%%%%
\begin{document}
\setlength{\droptitle}{-5em}    
%% %%%%%%%%%%%%%%%%%%%%%%%%%
\maketitle

\section*{Yλοποίηση ενός Γενετικού Αλγορίθμου για το Πρόβλημα του
Χρονοπρογραμματισμού Προσωπικού WHPP}
% %%%%%%%%%%%%%%%%%%%
{\bfseries \\\\Α) Εισαγωγή.}
\\\\Στο δεύτερο μέρος της 1ης προργαμματιστικής εργασίας του μαθήματος, κληθήκαμε να υλοποιήσουμε έναν γενετικό αλγόριθμο για το πρόβλημα του χρονοπρογραμματισμού του προσωπικού ενός νοσοκομείου, για ένα διάστημα 14 ημερών και 30 εργαζομένων.\\\\ Αρχικά περιγράφονται όλα τα βασικά κομμάτια του γενετικού μας αλγορίθμου και οι σχεδιαστικές μας επιλογές. Στη συνέχεια ακολουθεί σχολιασμός των αποτελεσμάτων και ιδέες για βελτίωση του προγράμματος.\\\\

{\bfseries Β) Γενετικός Αλγόριθμος.}
\\\\Όλοι οι γενετικοί αλγόριθμοι ακολουθούν έναν βασικό σκελετό υλοποιήσης(Figure 1). Επί γραμματικά, τα βήματα είναι τα εξής:\\\\
1) Δημιουργία ενός "ψευδό"-τυχαίου πληθυσμού.\\\\
2) Έλεγχος συνέπειας.\\\\
3) Αξιολόγηση αντικειμενικής συνάρτησης.\\\\
4) Ικανοποιήση των κριτηρίων βελτιστοποιήσης ?\\\\\\\\
Σε αυτό το σημείο, αν τα κριτήρια δεν ικανοποιούνται, συνεχίζουμε με την δημιουργία ενός νέου πλυθησμού, στον οποίο εφαρμόζονται οι εξής γενετικοί παράγωντες:\\\\
\boldsymbol{α) Selection} \\\\
\boldsymbol{β) Crossover}\\\\
\boldsymbol{γ) Mutation}\\\\
\boldsymbol{δ) Check}\\\\
Επσιστροφή πίσω στο βήμα 3.\\\\
Όταν και εφόσον τα κριτήρια στο βήμα 4 ικανοποιούνται, συνεχίζουμε στην εξαγωγή του αποτελέσματος.\\\\
Σε γενικές γραμμές, αυτά είναι τα βήματα για τον σχεδιασμό ενός γενετικού αλγορίθμου. Στη συνέχεια της αναφοράς, ακολουθεί αναλυτική περογραφή για κάθε βήμα.\\\\

\begin{figure}[!h]
    \centering
    \includegraphics[width=9cm]{images/image_1.png}
    \caption{Genetic Algorithm}
    \label{fig:my_label}
\end{figure}

{\bfseries Γ) Αρχικοποιήση πληθυσμού.}\\\\
Με τον όρο πληθυσμό, χαρακτηρίζεται το πλήθος των λύσεων του προβλήματος μας ή αλλιώς το πλήθος των προγραμμάτων που περνάνε τους αυστηρούς ελέγχους(Hard Constraints). Το πλήθος των χρωμοσωμάτων(δηλαδή ο πληθυσμός) έχουν δυο κύριους τρόπους αρχικοποιήσης.\\\\
1) Τυχαία αρχικοποιήση(η επιλογή μας).\\\\
2) Ευριστική αρχικοποιήση.\\\\
Μέσω της βιβλιοθήκης numpy και της συνάρτησης randint που εγκαταστήθηκε για την python, και για ένα άνω "φράγμα" αριθμού προγραμμάτων(population size=3000/2000), δημιουργήθηκε ένα τυχαίος αριθμός χρωμοσωμάτων, τα οποία περνάνε τους αυστηρούς περιορισμούς(όχι πάντα).\\\\
Στην αμέσως επόμενη γενιά, ο αριθμός των χρωμοσωμάτων είναι πιο κοντά σε αυτόν της αρχικοποιήσης, και αυτό οφείλεται στις διαδικασίες επιλογής-διαστεύρωσης-μετάλλαξης.\\\\
Σε ότι αφορά το πληθυσμιακό μας μοντέλο, επιλέχτηκε ενα δυναμικό μοντέλο το οποίο σε κάθε γενιά αλλάζει τον αριθμό των χρωμοσωμάτων που περνάνε στην επόμενη. Για να αποσαφηνιστεί αυτό, στην συνέχεια της αναφοράς, ακολουθεί το κριτήριο τερματισμού του αλγορίθμου μας. Να τονιστεί σε αυτό το σημείο, ότι για την συνέπεια παραγώμενω χρωμοσωμάτων, η συνάρτηση feasibility() ακολουθεί αυστηρά τους περιορισμούς της εκφώνησης που βρίσκονται στην σελίδα 15.\\\\

{\bfseries Δ) Κριτήριο Τερματισμού.}\\\\
Ως κριτήριο τερματισμού του αλγορίθμου εκ πρώτης όψης είναι ο αριθμός των επαναλήψεων που ορίσαμε με βάση την εκτέλεση του αλγορίθμου ως 400 γενιές. Ωστόσο, το πραγματικό κριτήριο είναι ο μηδενισμός των χρωμοσωμάτων που διέρχονται στην επόμενη γενιά. Πιο συγκεκριμένα, καθώς επιλέχτηκε ανακατανομή του πληθυσμού από γενιά σε γενιά, με τον μηδενισμό των χρωμοσωμάτων, καταλήγει το πρόγραμμα σε έναν μικρό αριθμό από βέλτιστες λύσεις(με την καλύτερη ποινή βάση των hard και soft constraints). \\\\
{\bfseries Ε) Επιλογή.}\\\\
Η επιλογή, είναι η διαδικασία επιλογής γονέων που "ταιριάζουν" και συνδυάζονται για την αναπαραγωγή της επόμενης γενιάς χρωμοσωμάτων. Η επιλογή του αλγορίθμου υλοποιήσης, είναι ζωτικής σημασίας για την σύγκλιση του αλγορίθμου σε ένα βέλτιστο αποτέλεσμα. Η συγκεκριμένη υλοποιήση, έγινε ακολουθώντας τον αλγόριθμο του τουρνουά. Η επιλογή αυτή οφείλεται στο γεγονός, ότι είναι αρκετα διαδεδωμένη τεχνική στην βιβλιογραφία που υπάρχει, ενώ παράλληλα είναι αρκετά εύκολη η υλοποιήση του αλγορίθμου.\\ Βασική ιδέα, είναι η τυχαία επιλογή χρωμοσωμάτων, και η σύγκριση αυτών για την εύρεση του "καλύτερου" γονέα(Figure 2).\\\\

\begin{figure}[!h]
    \centering
    \includegraphics[width=7cm]{images/image_2.png}
    \caption{Tournament selection}
    \label{fig:my_label}
\end{figure}

{\bfseries ΣΤ) Διασταύρωση.}\\\\
Εννοιολογικά, είναι η ίδια διαδικασία με την φυσική διασταύρωση δυο οργανισμών. Πρόκειται για την αναπαραγωγή ενός παιδιού από το γενετικό υλικό δυο γονέων. Ζητούμενο της εργασίας είναι η δημιουργία δυο τελεστών διασταύρωσης. Επιλέχτηκε η διασταύρωση ενός σημείου (one point) και η διασταύρωση πολλών σημείων(five points).\\
Για τον 1ο τελεστή(Figure 3-4), τυχαία επιλέγεται το σημείο διασταύρωσης των γονέων, και στη συνέχεια γίνεται η "ένωση" του "γενετικού υλικού".\\

\begin{figure}[!h]
\centering
\begin{minipage}{.5\textwidth}
  \centering
  \includegraphics[width=5cm]{images/image_3.png}
  \caption{Parents}
  \label{fig:my_label}
\end{minipage}%
\begin{minipage}{.5\textwidth}
  \centering
  \includegraphics[width=5cm]{images/image_4.png}
  \caption{1-point crossover Child}
  \label{fig:my_label}
\end{minipage}
\end{figure}
\\Αντίστοιχη διαδικασία, ακολουθεί και ο five-point crossover operator(Figure 5-6) αλλά για 5 τυχαία σημεία.\\

\begin{figure}[!h]
\centering
\begin{minipage}{.5\textwidth}
  \centering
  \includegraphics[width=5cm]{images/image_5.png}
  \caption{Parents}
  \label{fig:my_label}
\end{minipage}%
\begin{minipage}{.5\textwidth}
  \centering
  \includegraphics[width=5cm]{images/image_6.png}
  \caption{Multi-point crossover Child}
  \label{fig:my_label}
\end{minipage}
\end{figure}

{\bfseries Ζ) Μετάλλαξη.}\\\\
Ο δεύτερος από τους δύο γενετικούς παράγοντες, και ο τρόπος για να ξεφεύγουμε από τοπικά ακρότατα, η μετάλλαξη αποτελεί μια τυχαία επέμβαση σε κάποιο χρωμόσωμα για την εύρεση κάποιας άλλης λύσης. Η μετάλλαξη αφορά το κομμάτι ανακάλυψης του χώρου καταστάσεων και αποτελεί και αυτή ζωτικής σημασία παράγοντα για την σύγκλιση σε κάποιο αποτέλεσμα.\\
Υλοποιήθηκαν δύο αλγόριθμοι μετάλλαξης. O Random reseting και ο swap mutation.\\\\
Ο πρώτος αποτελεί μια διεύρυνση του bit flip mutation algorithm, για αναπαράσταση ακεραίων. Συγκεκριμένα, επιλέγει τυχαία κάποιο στοιχείο του χρωμοσώματος και το αλλάζει τιμή σύμφωνα με τον εξής μετασχηματισμό:\[0->3 ,\ 1->2 ,\ 2->1 ,\ 3->0\]
\\\\ O Swap mutation(Figure 7), με την σειρά του επιλέγει τυχαία δύο στοιχεία του χρωμοσώματος και τα αλλάζει θέση.
\begin{figure}[!h]
    \centering
    \includegraphics[width=7cm]{images/image_7.png}
    \caption{Swap mutation}
    \label{fig:my_label}
\end{figure}

{\bfseries H) Πειραματικά Αποτελέσματα.}\\\\
Η εκτέλεση της επιλογής νέου πληθυσμού, καθώς και των γενετικών τελεστών, πραγματοποιείται με κάποια πιθανοτικά κριτήρια. Συγκεκριμένα, η πιθανότητα επιλογής και διασταύρωσης πρέπει να είναι αρκετά μεγάλη έτσι ώστε να παρθεί το καλύτερο δυνατό αποτέλεσμα, ενώ η πιθανότητα μετάλλαξης πρέπει να είναι αρκετά μικρή, όχι όμως μηδενική, καθώς αυτό θα σημαίνει πρόορη σύγκλιση του αλγορίθμου.\\\\
Μετά από συνεχείς εκτελέσεις του προγράμματος για τυχαία φράγματα πληθυσμού, πιθανότητας επιλογής-διασταύρωσης-μετάλλαξης το ιδανικό που επιλέχτηκε ήταν:\\\\
1) population size = 3000 και 2000\\
2) Probability selection = 0.85\\
3) Probability crossover = 0.85\\
4) Probability mutation = 0.05\\\\
Δεν δίνεται κάποιο άλλο σετ τιμών, καθώς οποιοδήποτε σετ πιθανοτήτων με μικρή απόκλιση από αυτές τις τιμές έχει σχεδόν πανομοιότυπα αποτελέσματα.\\\\

{\bfseriesΑκολουθούν τα αποτελέσματα σύμφωνα με το ζητούμενο της εκφώνησης.}\\\\

1) tournament selection - one point crossover - random reseting - population = 2000.\\\\
\tab a) Best = 57435, Avg = 54456\\
\tab b) Best = 68323, Avg = 65695\\
\tab c) Best = 62042, Avg = 59380\\
\tab d) Best = 58931, Avg = 58094\\
\tab e) Best = 64931, Avg = 61367\\
\tab \boldsymbol{AvgBest:} = 62332.4\\\\\
2) tournament selection - one point crossover - swap mutation - population = 2000.\\\\
\tab a) Best = 62834, Avg = 62834\\
\tab b) Best = 55742, Avg = 55742\\
\tab c) Best = 63240, Avg = 62881\\
\tab d) Best = 60740, Avg = 60740\\
\tab e) Best = 67243, Avg = 65397\\
\tab \boldsymbol{AvgBest:} = 61959.8\\\\\
3) tournament selection - one point crossover - random reseting - population = 3000\\\\
\tab a) Best = 62438, Avg = 62438\\
\tab b) Best = 58733, Avg = 58733\\
\tab c) Best = 71227, Avg = 68850\\
\tab d) Best = 53837, Avg = 53435\\
\tab e) Best = 61043, Avg = 56464\\
\tab \boldsymbol{AvgBest:} = 61455.6\\\\\
4) tournament selection - one point crossover - swap mutation - population = 3000\\\\
\tab a) Best = 60731, Avg = 58051\\
\tab b) Best = 70432, Avg = 61082\\
\tab c) Best = 69237, Avg = 65095\\
\tab d) Best = 56731, Avg = 56731\\
\tab e) Best = 65237, Avg = 65237\\
\tab \boldsymbol{AvgBest:} = 64473.6\\\\\
5) tournament selection - five point crossover - random reseting - population = 3000\\\\
\tab a) Best = 64832, Avg = 61523\\
\tab b) Best = 69834, Avg = 67623\\
\tab c) Best = 66943, Avg = 62482\\
\tab d) Best = 62524, Avg = 60849\\
\tab e) Best = 75630, Avg = 68815\\
\tab \boldsymbol{AvgBest:} = 67952.6\\\\\
6) tournament selection - five point crossover -  swap mutation - population = 3000\\\\
\tab a) Best = 51644, Avg = 51644\\
\tab b) Best = 66725, Avg = 66725\\
\tab c) Best = 49244, Avg = 49244\\
\tab d) Best = 61741, Avg = 60774\\
\tab e) Best = 59134, Avg = 58909\\
\tab \boldsymbol{AvgBest:} = 57697.6\\\\\
7) tournament selection - five point crossover - swap mutation - population = 2000\\\\
\tab a) Best = 62039, Avg = 59261\\
\tab b) Best = 61238, Avg = 59711\\
\tab c) Best = 62151, Avg = 58309\\
\tab d) Best = 65230, Avg = 60194\\
\tab e) Best = 61644, Avg = 57431\\
\tab \boldsymbol{AvgBest:} = 62460.4\\\\\
8) tournament selection - five point crossover - random reseting - population = 2000\\\\
\tab a) Best = 63373, Avg = 62155\\
\tab b) Best = 66831, Avg = 63107\\
\tab c) Best = 57126, Avg = 57126\\
\tab d) Best = 59035, Avg = 58868\\
\tab e) Best = 74734, Avg = 70099\\
\tab \boldsymbol{AvgBest:} = 64219.8\\\\\

{\bfseries Γραφήματα καλύτερων περιπτώσεων.}\\\\
Ακολοθούν οι τίτλοι και τα γραφήματα\\
1) tournament - one point crossover - random reseting - population = 3000\\
2) tournament - one point crossover - swap mutation - population = 3000\\
3) tournament - one point crossover - random reseting - population = 2000\\
4) tournament - one point crossover - swap mutation - population = 2000\\
5) tournament - multi point crossover - random reseting - population = 3000\\
6) tournament - multi point crossover - swap mutation - population = 3000\\
7) tournament - multi point crossover - random reseting - population = 2000\\
8) tournament - multi point crossover - swap mutation - population = 2000\\
\begin{figure}[!h]
\centering
\begin{minipage}{.5\textwidth}
  \centering
  \includegraphics[width=7cm]{images/1_1.png}
  \caption{Best score(1)}
  \label{fig:my_label}
\end{minipage}%
\begin{minipage}{.5\textwidth}
  \centering
  \includegraphics[width=7cm]{images/1_2.png}
  \caption{Average Score(1)}
  \label{fig:my_label}
\end{minipage}
\end{figure}
\begin{figure}[!h]
\centering
\begin{minipage}{.5\textwidth}
  \centering
  \includegraphics[width=7cm]{images/1_3.png}
  \caption{Best score(2)}
  \label{fig:my_label}
\end{minipage}%
\begin{minipage}{.5\textwidth}
  \centering
  \includegraphics[width=7cm]{images/1_4.png}
  \caption{Average Score(2)}
  \label{fig:my_label}
\end{minipage}
\end{figure}
\begin{figure}[!h]
\centering
\begin{minipage}{.5\textwidth}
  \centering
  \includegraphics[width=7cm]{images/1_5.png}
  \caption{Best score(3)}
  \label{fig:my_label}
\end{minipage}%
\begin{minipage}{.5\textwidth}
  \centering
  \includegraphics[width=7cm]{images/1_6.png}
  \caption{Average Score(3)}
  \label{fig:my_label}
\end{minipage}
\end{figure}
\begin{figure}[!h]
\centering
\begin{minipage}{.5\textwidth}
  \centering
  \includegraphics[width=7cm]{images/1_7.png}
  \caption{Best score(4)}
  \label{fig:my_label}
\end{minipage}%
\begin{minipage}{.5\textwidth}
  \centering
  \includegraphics[width=7cm]{images/1_8.png}
  \caption{Average Score(4)}
  \label{fig:my_label}
\end{minipage}
\end{figure}
\begin{figure}[!h]
\centering
\begin{minipage}{.5\textwidth}
  \centering
  \includegraphics[width=7cm]{images/1_9.png}
  \caption{Best score(5)}
  \label{fig:my_label}
\end{minipage}%
\begin{minipage}{.5\textwidth}
  \centering
  \includegraphics[width=7cm]{images/1_0.png}
  \caption{Average Score(5)}
  \label{fig:my_label}
\end{minipage}
\end{figure}
\begin{figure}[!h]
\centering
\begin{minipage}{.5\textwidth}
  \centering
  \includegraphics[width=7cm]{images/1_01.png}
  \caption{Best score(6)}
  \label{fig:my_label}
\end{minipage}%
\begin{minipage}{.5\textwidth}
  \centering
  \includegraphics[width=7cm]{images/1_02.png}
  \caption{Average Score(6)}
  \label{fig:my_label}
\end{minipage}
\end{figure}
\begin{figure}[!h]
\centering
\begin{minipage}{.5\textwidth}
  \centering
  \includegraphics[width=7cm]{images/1_03.png}
  \caption{Best score(7)}
  \label{fig:my_label}
\end{minipage}%
\begin{minipage}{.5\textwidth}
  \centering
  \includegraphics[width=7cm]{images/1_04.png}
  \caption{Average Score(7)}
  \label{fig:my_label}
\end{minipage}
\end{figure}
\begin{figure}[!h]
\centering
\begin{minipage}{.5\textwidth}
  \centering
  \includegraphics[width=7cm]{images/1_05.png}
  \caption{Best score(8)}
  \label{fig:my_label}
\end{minipage}%
\begin{minipage}{.5\textwidth}
  \centering
  \includegraphics[width=7cm]{images/1_06.png}
  \caption{Average Score(8)}
  \label{fig:my_label}
\end{minipage}
\end{figure}
\FloatBarrier
{\bfseries Θ) Σχολιασμός.}\\\\
Αρχικά οι ταλαντώσεις στα γραφήματα είναι αναμενώμενες, καθώς ο πληθυσμός αλλάζει διαρκώς, με αποτέλεσμα να αλλάζει και η καλύτερη τιμή διαρκώς. Η ταλαντώσεις πάλι της μέσης τιμής είναι απολύτος δικαιολογημένες, όπως επίσης και η αύξηση κάποιες φορές των τιμών, καθώς όταν το πλήθος αλλάζει, αλλάζει και ο παρονομαστής που ζυγίζει τον μέσο όρο.\\ Παρατηρείται γενικά ότι η swap μετάλλαξη φέρνει καλύτερα αποτελέσματα, όταν συνδυαστεί με five point crossover σε άνω φράγμα πληθυσμού 3000. Από την άλλη, η random reseting μετάλλαξη, συνδυάζεται καλύτερα με one point crossover και πάλι άνω φράγμα πληθυσμού 3000.\\\\
Είναι αναμενώμενο, για όσο μεγαλύτερο αριθμό χρωμοσωμάτων έχουμε, τόσο καλύτερα αποτελέσματα να παίρνουμε, καθώς ο αλγόριθμος συγκλίνει πιο αργά και ομαλά στην βέλτιστη λύση.\\\\
Σε ότι αφορά τους γενετικούς τελεστές, η σύγκριση τους δεν είναι δίκαιη, αν και γενικά όπου υπάρχει swap mutation τα αποτελέσματα είναι σχετικά καλύτερα.\\Για πιθανή βελτίωση του προγράμματος μας, θα μπορούσε να εξεταστεί και κάποιο άλλο είδος επιλογής ή από γενιά σε γενιά να διατηρηθεί σταθερό το πλήθος των χρωμοσωμάτων.\\Η προσέγγιση μας βασίστηκε σε papers σημειώσεις και forums τα οποία βρίσκονται στην ακόλουθη ενότητα.\\\\
{\bfseries Ι) Βιβλιογραφία.}\\\\
\url{https://www.researchgate.net/post/How_to_overcome_strong_local_minima_in_Genetic_Algorithm}\\\\

\url{https://www.researchgate.net/publication/325011628_Comparative_Study_of_Different_Selection_Techniques_in_Genetic_Algorithm}\\\\

\url{https://tik-old.ee.ethz.ch/file//6c0e384dceb283cd4301339a895b72b8/TIK-Report11.pdf}\\\\

\url{https://www.researchgate.net/post/Which_selection_technique_is_best_in_genetic_algorithm}\\\\

\url{https://www.tutorialspoint.com/genetic_algorithms/genetic_algorithms_population.htm}\\\\

\end{document}